\chapter{Dnevnik promjena dokumentacije}
		
		\textbf{\textit{Kontinuirano osvježavanje}}\\
				
		
		\begin{longtblr}[
				label=none
			]{
				width = \textwidth, 
				colspec={|X[2]|X[13]|X[3]|X[3]|}, 
				rowhead = 1
			}
			\hline
			\textbf{Rev.}	& \textbf{Opis promjene/dodatka} & \textbf{Autori} & \textbf{Datum}\\[3pt] \hline
			0.1 & Napravljen predložak.	& * & 25.10.2023. 		\\[3pt] \hline 
			0.2	& Dopisane upute za povijest dokumentacije.\newline Dodane reference. & * & 24.08.2013. 	\\[3pt] \hline 
			0.5 & Dodan \textit{Use Case} dijagram i jedan sekvencijski dijagram, funkcionalni i nefunkcionalni zahtjevi i dodatak A & * & 26.10.2023. \\[3pt] \hline 
			0.6 & Arhitektura i dizajn sustava, algoritmi i strukture podataka & * & 27.10.2013. \\[3pt] \hline 
			0.8 & Povijest rada i trenutni status implementacije,\newline Zaključci i plan daljnjeg rada & * & 2.11.2023. \\[3pt] \hline 
			0.9 & Opisi obrazaca uporabe & * & 07.11.2023. \\[3pt] \hline 
			0.10 & Preveden uvod & * & 10.11.2023. \\[3pt] \hline 
			0.11 & Sekvencijski dijagrami & * & 15.11.2023. \\[3pt] \hline 
			0.12.2 & Napravljen dijagrama razreda & * & 16.11.2023. \\[3pt] \hline 
			\textbf{1.0} & Verzija samo s bitnim dijelovima za 1. ciklus & * & 17.11.2023. \\[3pt] \hline 
		\end{longtblr}
	
	
		\textit{Moraju postojati glavne revizije dokumenata 1.0 i 2.0 na kraju prvog i drugog ciklusa. Između tih revizija mogu postojati manje revizije već prema tome kako se dokument bude nadopunjavao. Očekuje se da nakon svake značajnije promjene (dodatka, izmjene, uklanjanja dijelova teksta i popratnih grafičkih sadržaja) dokumenta se to zabilježi kao revizija. Npr., revizije unutar prvog ciklusa će imati oznake 0.1, 0.2, …, 0.9, 0.10, 0.11.. sve do konačne revizije prvog ciklusa 1.0. U drugom ciklusu se nastavlja s revizijama 1.1, 1.2, itd.}